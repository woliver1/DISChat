%% 
%
% -----< P R E A M B L E >-----
%
\documentclass[12pt]{article}
%
% ----- Packages
%
%\usepackage[utf8]{inputenc}
\usepackage{algorithm}
\usepackage{algpseudocode}
\usepackage{graphicx}
\usepackage{setspace}
\usepackage{minted}    % Source highlighing
\usepackage{longtable}
\usepackage{multirow}
%
% -----< E N D   P R E A M B L E >-----
%
%
% -----< B E G I N   D O C U M E N T >-----
%
\begin{document}
%
% ----- Define new commands here.
% 
\doublespacing
%
% ----- Start the frontmatter
%
\author{William Oliver}
%\date
\title{A Proposal for a simple DIS chat protocol.}
\maketitle
\tableofcontents
%\listoffigures
%\listoftables
%\listoflistings
%
% ----- Start of the Contents
%
\section{Introduction}
\label{sec:intro}
During the setup and conduct of distributed simulation exercises there
is often a need to communicate with the other participants. When the
other participants are remotly located there may be severe
restrictions on the network protocols permitted. Typically the DIS
network connectivity is available early in the network testing and
configuring other communications devices, VoIP, VTC and DIS radios can
take some time.  A simple DIS based chat system would allow simple
communication without requireing changes to the network design. Given
that DIS is, by its nature, distributed --- that is, without a client
server architecture, the chat client was designed to similarly be
serverless.

\section{Outline}
\label{sec:outline}

The broad outline is to use DIS PDUs to transmit chat messages. As DIS
is commonly used on multicast or broadcast networks there is no need
to have a server, clients simply broadcast messages and listen on the
broadcast address for messages.

To create a useful prototype in a reasonable time frame, and to allow
others to easily write their own implimentation\footnote{This is
  especially important in military simulations, as security
  restrictions may make it hard for third party software to be
  installed.} a very simple protocol was developed.

Chat clients show the user two things, a list of users and a message
window. Chat protocols also usually allow two users to talk to one
another, and to chat in a room. Three things then are required to be
communicated, a sender, a recipient --- chat rooms can be represented
by a recipient --- and the message. To keep the protocol simple we
have chosen to use only one kind of PDU, the comment PDU. A Variable
Datum Record can be used to encode the data we want, the message. The
sender and recipient can be efficiently represented as DIS entity ids,
these are already in the comment PDU for both the sender and
recipient. As a convienience we use a variable datum to put a human
readable username which is associated with the DIS id. For our proof
of concept we did not impliment any chat rooms, noting that there are
effectivly chat rooms, one for the all sites (0:0:0) one for the all
applications at a site (x:0:0) and one for all entities for an
application (x:y:0). The rules for sending and receiving chat messages
in/to chat rooms will be expanded in future versions.

\section{Design}
\label{sec:design}
To create a prototype application as quickly as possible OpenDIS was
used and useful code snippets were taken from another project.
another project.

\subsection{GUI}
\label{sec:gui}
A Graphical User Interface was creted using the GUI builder tool in
Netbeans --- a standard swing/awt program, shown below.

\begin{figure}[!hbp]
\includegraphics[scale=0.4]{graphics/gui.png}
\caption{DISChat Graphical User Interface.}
\end{figure}

\subsection{Software Design}
\label{sec:software-design}
The design was borrowed from another DIS application that was required
to handle a high load, the chat application in not anticipated to
handle high packet load, but it was easiest to leave the design as
is. The basic design is to use queues to move data between threads
that perform specific tasks. Two network threads are spawned, one for
receiving and one for sending. The receiving thread receives a packet
and puts it into a receive queue and immediately returns to waiting on
the network. Another thread reads from the receive queue and
determines if we received a valid comment pdu --- if yes, extract the
message and display it in the GUI if not, discard, then return to
monitoring the queue. The sending queue waits for the GUI to send it a
message, marshalls it, and sends it.

\subsubsection{Network Thread}
\label{sec:network-thread}

TODO - split the above section into these sub headings.

\subsubsection{PDU hadnling}
\label{sec:pdu-handling}

\section{Alternatives}
\label{sec:alternatives}

A number if alternative designs were considered.  First, using SISO-J
messages as Link 16 has a chat protocol which could be re-used. This
however, would be a complicated protocol, as you would need to both
decode the DIS protocol and then appropriate SISO-J/link messages.
%
\\
\textbf{pros} \\
potentially compatible with real tactical chat networks.
%
\\
\textbf{cons} \\
complicated implimentation, limited implimentation outside the air
domain.

We also considered implimenting chat using a transmitter pdu
to represent the client, and signal pdus for the actual communication,
similar to the SISO-J protocol above, but without the overhead of the link
protocol.
%
\\
\textbf{pros} \\
similar to the way other communictations in DIS works.
%
\\
\textbf{cons} \\
requires two pdus.

\section{Further Work}
\label{sec:further-work}
We have implemented only a very basic chat client, and simple issuance
and receipt rules. There are a number of ways we could extend the
utility of the system. One would be to impliment chat rooms. This
could be done by using the DIS entity id to create new rooms, a name
would need to be associated with the room id and the client would need
a way of adding that id as messages it wants to receive.

A further improvement would be the ability to transfer files. This
should be fairly simple to do, split a file into chuncks of $x$ bytes,
put them in a pdu with some form of sequence identifier ( as you would
need to reassemble in order). The hard bit would be implimenting a
protocol for requesting missing sections (UDP being an unreliable
protocol). It is worh noting that file transfer is equally faesable
for both plain text and binary as the Variable Datum Record is
agnostic to its contents

\section{User Guide}
\label{sec:user-guide}

\subsection{Launching the Client}
\label{sec:launching}

\begin{verbatim}
cd DISChat
java -jar target\DISChat.jar
\end{verbatim}

\subsection{Connecting the Client}
\label{sec:connecting}

\section{Issuance Rules}
\label{sec:issuance-rules}

Once a connection has been established (see
section\ref{sec:connecting}) the client sends a comment PDU with the
\emph{exercise id}, the source \emph{site, application} and
\emph{entity id} set and the \emph{variable datum}
600001\footnote{Please not that the datum ids of 600001 and 60002 hold
  no significance and should eventually be aggreeed upon, by
  convention at least} set to the users chosen \emph{username}. This
information is then sent as a heartbeat even when no messages are
being exchanged, this allows the other chat clients to build up a list
of chat users. The heartbeat in the reference implimentation is 5
seconds. No other Datum Record or Vaiable Datum Record is to be set in
the PDU.

When a message is ready to send, a comment PDU containing the
\emph{exercise id}, the source and destiation \emph{site, application
  and entity id}s, the \emph{variable datum} 600001 set to the users
chosen \emph{username} and the the \emph{variable datum} 600002
containing the message, is sent. No other Datum Record or Vaiable
Datum Record is to be set in the PDU.

In this version of the chat client the destination address makes
\emph{no} difference. All client should listen to all
messages.\footnote{In future versions the ability to specify
  indiviauals (\emph{entity id}), sites (\emph{site id}) and other
  combinations will be made available. At present this is just a proof
  of concept.}


\section{Receipt Rules}
\label{sec:receipt-rules}

On receipt of a comment PDU the presence of \emph{Variable Datum
  Record} 600001 indicates that the PDU is a chat PDU, the presence of
both a 600001 and 600002 record indicates that it is a chat PDU with a
message.  No other Datum or Variable Datum Records should be present in
the PDU. If there are, they are to be ignored, and optionally the PDU
is to be logged as malformed. The \emph{exercise id} and the \emph{source id} are to
be extracted, along with the \emph{username} (the 600001 variable datum). If
the \emph{exercise id} matches that of the active exercise the \emph{source id} with
its associated username is to be added to a list representing all chat
users in the current exercise.  An exercise id of 0 is to be
interpreted as meaning that the message is for everyone regardless of
which exercise thay are participating in.



If a 600002 variable datum is present as well as the username, then
the message is to be extracted and displayed. It is recommended that
the standard chat convention of displaying the senders \emph{username}
prefixed to the message be observed as this helps identify who the
message is from\footnote{DIS ids not being the easiest for people to
  identify.}. In our implimentation the convention is to display the
DIS id with the username in parantheses, followed by a colon, a space
and the message, as shown.
\begin{verbatim}
1:2:3 (username): Message text goes here.
\end{verbatim}

\appendix

\section{Structure of the Comment PDU}

\begin{table}
  \centering
  \begin{tabular}{l|l|l }
    %\multicolumn{2}{c}{Comment PDU} \\
    %\hline
    %\endfirsthead
    %Record Name & Field Name \\
    %\hline \hline
    %\endhead
    \hline
    Record Name                      &                                              & Field Name                       \\ \hline \hline
    \multirow{8}{*}{PDU Header}      &                                              & Protocol Version                 \\
                                     &                                              & Exercise ID                      \\
                                     &                                              & PDU Type                         \\
                                     &                                              & Protocol Family                  \\
                                     &                                              & TimeStamp                        \\
                                     &                                              & Length                           \\                    
                                     &                                              & PDU Status                       \\                 
                                     &                                              & Padding                          \\ \hline
    \multirow{3}{*}{Originating ID}  &                                              & Site Number                      \\
                                     &                                              & Application Number               \\
                                     &                                              & Reference Number                 \\ \hline
    \multirow{3}{*}{Receiving ID}    &                                              & Site Number                      \\
                                     &                                              & Application Number               \\
                                     &                                              & Reference Number                 \\ \hline
    \parbox[t]{2mm}{\multirow{11}{*}{\rotatebox[origin=c]{90}{Datum Specification Record}}} & & Number of Fixed Datum Records    \\ 
                                     &                                              & Number of Variable Datum Records \\ \cline{2-3}
                                     & \multirow{4}{*}{\textit{Variable Datum \#1}} & Variable Datum ID                \\
                                     &                                              & Variable Datum Length            \\
                                     &                                              & Variable Datum Value             \\
                                     &                                              & Padding to 64-bit boundary       \\ \cline{2-3}
                                     & \multicolumn{1}{c|}{$\vdots$}               & \multicolumn{1}{c}{$\vdots$}   \\    \cline{2-3}
                                     & \multirow{4}{*}{\textit{Variable Datum \#N}} & Variable Datum ID                \\
                                     &                                              & Variable Datum Length            \\
                                     &                                              & Variable Datum Value             \\
                                     &                                              & Padding to 64-bit boundary       \\
    \hline
  \end{tabular}
  \label{tbl:dis-comment-pdu}
  \caption{Comment PDU} 
\end{table} 
\section{Testing Stuff}

\begin{algorithm}
\caption{My algorithm}\label{euclid}
\begin{algorithmic}[1]
\Procedure{CH\textendash Election}{}
\State Broadcast HELLO message to its neighbor
\EndProcedure
\end{algorithmic}

\end{algorithm}

\section{Pseudocode}

\begin{algorithm}
  \caption{The \texttt{main} procedure of DISChat.}\label{alg:chatgui}
  \begin{algorithmic}[1]
    \Procedure{DISChat Main Procedure}{}
    \State Constructor $\to$ new ChatController(chatTextArea, usersTextArea)
    \State main $\to$ init GUI
    \Loop
    \State Check for input
    \If {input is button press}
    \State Call button action
    \ElsIf {input is value entry}
    \State Update value based on entry
    \EndIf
    \EndLoop
    \EndProcedure
  \end{algorithmic}
\end{algorithm}

\begin{algorithm}
  \caption{The \texttt{ChatController} logic.}\label{alg:chatcontroller}
  \begin{algorithmic}[1]
    \Procedure{ChatController Object}{}
    \State $pduQueue \gets LinkedBlockingQueue$
    \State $sendQueue \gets LinkedBlockingQueue$
    \State $disChatReceiver \gets DisChatReceiver$
    \State $disChatThread \gets disChatReceiver\; as\; Thread$
   
    \State $senderId \gets EntityID$
    \State $receiverId  \gets EntityID$
    \State $userList \gets DisChatUserList$
    \State $disChatThread \to start$
    \EndProcedure
  \end{algorithmic}
\end{algorithm}

\begin{algorithm}
  \caption{The \texttt{connect} button logic.}\label{alg:connectbutton}
  \begin{algorithmic}[1]
    \Procedure{ChatController Object}{}
    \State $senderId \gets$ GUI sender value
    \State $receiverId \gets$ GUI receiver value
    \State $username \gets$ GUI uesrname value
    \State $exerciseid \gets$ GUI exercise ID value
    \State $ip \gets$ GUI IP address value
    \State $port \gets$ GUI Port value
    \If {$state_{button}$ is selected}  
      \State $success \gets (chatcontroller \to connectsocket$ with $ip$, $port)$
      \If {$success$ is $true$}
        \State $button\; text \gets $ Disconnected
      \Else
        \State $button\; text \gets $ Connect \Comment{Set button to initial state to note we have failed}
      \EndIf
    \EndIf
    \EndProcedure
  \end{algorithmic}
\end{algorithm}

\begin{algorithm}
  \caption{The DisReceiver}\label{alg:disreceiver}
  \begin{algorithmic}[1]
    \Procedure{DIS Receiver}{}
    \State $init pduQueue$
    \State $packet \gets (socket \to receive)$
    \If {$(packet \to data)$ is $DIS PDU$}
      \State $disobject \gets pdu$ and $time$
      \State $disobject \to pduQueue$
    \State 
    \EndProcedure
  \end{algorithmic}
\end{algorithm}

\section{Listings}
% Define background colour to make listings stand out a bit.
\definecolor{bg}{rgb}{0.95,0.95,0.95}
%\begin{listing}[H]
%  \inputminted[linenos,bgcolor=bg]{java}{../src/main/java/woliver/dischat/ChatController.java}
%  \caption{ChatController Class.}
%  \label{lst:chatcontroller}
%\end{listing}

\begin{listing}[H]

   \begin{minted}[linenos,fontsize=\footnotesize]{java}
public void sendButtonPressed(String message) {
    logger.entry(message);
    DisChatMessage chatmsg;
    chatmsg = new DisChatMessage();
    if (message.trim().length() > 0)  {
        chatmsg.setMessage(message);
        chatmsg.setSenderId(this.senderId);
        chatmsg.setReceiverId(this.receiverId);
        chatmsg.setSendername(this.username);
        chatmsg.setExerciseId(this.exerciseId);
        logger.debug("Send button pressed - message is {}", message);
        messageReceived(chatmsg); /* Sends message to message area */

        try {
            this.sendQueue.put(chatmsg);
        } catch (InterruptedException ex) {
            logger.error("FIXME - unable to queue message!", ex);
        }
    } else {
        logger.warn("Invalid message --- ({}).", message);
    }
    logger.exit();
}
   \end{minted}
   \caption{Java method to respond to Send button}
   \label{lst:awk-script}
\end{listing}
\end{document}
